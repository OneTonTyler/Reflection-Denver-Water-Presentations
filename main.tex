\documentclass[12pt]{article}
\usepackage[utf8]{inputenc}

\usepackage[margin=1in]{geometry}
\usepackage{lipsum}

\usepackage[backend=biber,style=ieee]{biblatex}
\addbibresource{sources.bib}

\usepackage{titling}
\newcommand{\subtitle}[1]{%
	\posttitle{%
		\par\end{center}
	\begin{center}\large#1\end{center}
	\vskip0.1em}}%

\title{Reflection\\
Denver Water presentations}
\subtitle{PEGN 430A}
\author{Tyler Singleton}
\date{10 March 2022}

\begin{document}
\maketitle

\newpage
\setlength{\parindent}{0pt}

% --- Questions Section --- %
\textbf{Julia Keedy's Presentation} \\

% Question 1
\textbf{1. If the theory of water supply modeling is so simple, what are some of the aspects that make it so complex?} \\

As mentioned by Julia Keedy, water supply and modeling is simple. The equations follow a basic concept of conservation as what goes out of the system (expenses) must be equal to the system inputs (income) else if the expenses exceed the income, our virtual water back account balance will result in a deficit. However, while the equations may be simple, the system itself is complex. Some of the aspects that make water supply modeling so complex range from determining which model to use, legal obligations, how many years into the future does a client need water, local demand changes, and changes in global climate. \\

% Question 2
\textbf{2. If you were the water supply modeler for a large water utility, how would you recommend planning for an uncertain future (e.g. what/how would you model to account for uncertainty)? (adapted from one of Taylor Winchell’s questions below).} \\

Having assisted in planning and logistics before, our strategy was to determine the minimum requirements then always add 20\% to the total. This was especially critical for fuel. A surplus is never an issue. The excess fuel can be sold back or used for future operations. However a shortage is devastating. If I were to supply a model for a larger water utility, I would recommend to adding 20\% to the anticipated demand. \\

% Question 3
\textbf{3. Did water supply modeling create the problems on the Colorado River? Can it fix the problems on the Colorado River?} \\

I do not think modeling water supply created the problems on the Colorado River, rather it created overconfidence. I think modeling could also fix the problem. Firstly, models are useful to conceptualize a general trend and allows us to plan effectively. We are also in a new era of machine learning that may allow for more accurate predictions. \\

\textbf{Taylor Winchell's Presentation} \\

% Question 4
\textbf{4. List six different ways that a water utility could be impacted by climate change and discuss three of them.} \\

\begin{enumerate}
    \item Underground aquifers take longer to recharge.
    \item Demand increases as more water is needed for agriculture.
    \item Lakes become hotter making it less habitable for aquatic life.
    \item Reduced snow pack.
    \item Reduced rainfall.
    \item More extreme weather.
\end{enumerate}

Discussing (3), hotter lakes will create significant algae blooms that greatly impact water quality and water production rates. Algae removes oxygen from the lake which slowly makes it less habitual for aquatic life which helps to maintain the water quality. A drop in quality also makes production slower and the filters used to clean will need a higher frequency of maintenance. \\

Item (5), reduced rain fall will cause significant periods of droughts and consequently give a drier layer of topsoil. Dry topsoil does not allow water to penetrate into the subsurface effectively. Thus when it does rain, most of the rain may run off the surface into streams and rivers instead of recharging our underground aquifers from which we utilize for public waters. \\

Lastly (6), extreme weather places a larger burden on our wastewater management systems and water collection systems. Periods of drought followed by periods of significant rainfall will place an unnecessary strain on these systems. \\

% Question 5
\textbf{5. If you were CEO of a large Western U.S. water utility, what steps would you take to adapt to a changing climate?} \\

If I were CEO of this large water utility, I would increase the cost of water. This follows the basic concept of supply and demand. Larger costs will produce lower demand. However, this will disproportionately affect those experiencing hardships, so a subsidy will need to be considered. \\

% Question 6
\textbf{6.  What role can the water industry play in mitigating greenhouse gases? And how does that relate to climate adaptation?} \\

The role of the water industry can play in greenhouse gas mitigation would be to promote ways in which everyone can reduce their carbon footprint. Additionally, water purification plants require a lot of energy, that could come from renewable resources such as wind and solar. This relates to climate adaption by reducing our overall impact to prevent further damaging our climate. 

\end{document}
